% !TeX spellcheck = en_US
% !TeX encoding = UTF-8
% !TeX root = ../document.tex

\chapter{Conclusion}
In this thesis, several new features of the \ac{MUSiC} analysis have been introduced and evaluated with regard to the discovery potential towards new physics. 

The first new contribution within the existing analysis was to include \Pqb-tagged jets as dedicated physics objects, which has not been attempted since the increase of energy to $\sqrt{s} = \SI{13}{\TeV}$ in 2015. Although the increase in sensitivity could not be explicitly shown, this feature should be reevaluated in further studies and is completely implemented.

On the technical side of the analysis, the growing complexity of the tool chain, from classification to statistical inference, has motivated employing further automation. Additionally, a \acl{LUT} has been implemented and evaluated in order to optimize the performance of the automated search for deviations.

Taking up discussions that have been published in a thesis eight years ago\cite{Schmitz:ModelUnspecificSearch}, the potential of a log-normal prior within the local test statistic has been discussed. For this purpose, a study of the coverage behavior of both the normal- and log-normal prior has been performed, indicating that the log-normal case features superior coverage properties. However, several difficulties regarding pseudo-experiments with a log-normal prior arise. Possible mitigations have been discussed, and it was decided to maintain using the normal prior with additional constraints on the search space.

Another new feature within the analysis is the computation of a global $p$-value. This allows to quantify deviations within the distribution of \ptilde-values, instead of judging deviations by eye. The feature enables future analysts to observe whether a certain change of a parameter value or the implementation of a new feature increases the sensitivity towards certain benchmark models.

Finally, the sensitivity of the analysis towards four models of new physics has been assessed. These benchmark models have been chosen to cover a wide range of possible signatures of new physics. By combining simulated events of these models with \acl{SM} processes, generating pseudo-experiments and analyzing the resulting distributions with the automated search, the impact of each model on the statistical inference has been evaluated. 
In two instances, the \PWprime and the Seesaw Type-III model, the sensitivity did not suffice to discover the presence of the simulated events. In these cases, a comparison to the dedicated analyses was drawn, yielding several suggestions for features that may increase the sensitivity.
The other two models, semiclassical and quantum black holes, could successfully discovered. The highest discoverable object masses are in agreement with mass limits from corresponding dedicated analyses.

In the future, we would like to encourage analysts who pursue a model independent analysis approach to regularly reevaluate the sensitivity of their analysis to a wide range of new physics simulations. Furthermore, we would suggest to refine the idea of a global $p$-value and a measure of sensitivity, as informed decisions will help the analysis to remain lean and efficient for discovering new physics.

%automation
%bJets
%LUT
%coverage
%evaluation of lognormal
%phat
% !TeX spellcheck = de_DE
% !TeX encoding = UTF-8
% !TeX root = ../document.tex

\vspace{1cm}
{\usekomafont{chapter}Kurzdarstellung}
\chapterheadendvskip

\begin{otherlanguage}{german}
In den Jahren 2015 und 2016 wurden vom \acs{CMS}-Experiment Kollisionen von Protonenpaaren bei einer Schwerpunktsenergie von $\sqrt{s} = \SI{13}{\TeV}$ beobachtet. Mithilfe der modellunabhängigen Suche in \acs{CMS} (\ac{MUSiC}) können die aufgezeichneten Daten automatisiert nach Anzeichen neuer Physik durchsucht werden.

Die \ac{MUSiC} Analyse besteht aus einem mehrstufigen Verfahren, bei dem die aufgezeichneten Ereignisse zuerst ihrem Endzustand gemäß in verschiedene Klassen eingeteilt werden. In jeder Klasse werden daraufhin Histogramme einiger kinematischer Variablen aggregiert und die größte Abweichung zwischen der beobachteten Verteilung und einer Erwartung aus \acl{MC}-Simulationen des Standardmodelles berechnet. Im Rahmen eines statistischen Hypothesentests wird schließlich festgestellt werden, ob beobachtete Daten Hinweise auf neue Physik beinhalten.

In dieser Arbeit wird das Entdeckungspotential der Analyse untersucht. Zu diesem Zweck wird zuerst eine geeignete quantitative Teststärke definiert. Diese wird anschließend für Simulationsergebnisse vier verschiedener Modelle neuer Physik ausgerechnet. Neben dem absoluten Wert in Bezug auf diese Modelle wird der Einfluss verschiedener Funktionalitäten und Parameter auf die Sensitivität untersucht.
\end{otherlanguage}
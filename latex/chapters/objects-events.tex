% !TeX spellcheck = en_US
% !TeX encoding = UTF-8
% !TeX root = ../document.tex

\chapter{Event and Object Selection}
\label{chap:selection}

\section{Event Selection}
\subsection{Triggers}
\subsection{MET-Filters}
Possible reasons for excluding triggered events from the analysis are e.g.
reconstruction problems or saturation of the tracker by calibration layers.


\section{Object Selection}
\subsection{Offline Reconstruction}
\subsection{Identification}
\subsection{b/t-Tagging}
\label{sec:b_tagging}

Tracking resolution 20um transverse, 30um along beam axis, efficiency 98\%
long life time of b hadrons
displacement to primary vertex

stable particles from PF
jets: charged energy from tracker + ECAL/HCAL, neutral: ECAL/HCAL
anti-kT, R=4 (AK4) jets
CSV=combined secondary vertex, new version CSVv2
multivariate combination of displaced tracks with info on secondary vertices
three categories based on number of secondary (or alternatively "pseudo")
vertices
perceptron with one hidden layer calculates discriminating variable (Run I: likelihood ration, bad for correlations)
likelihood ratio combines categories

operating point CSVv2T (tight, discriminator=0.935): eff $\approx \SI{49}{\percent}$, misstag rate between 0 and 0.01 percent


\cite{CMS:CMS-PAS-BTV-15-001},

Scale Factor uncert 2015 \cite[Fig. 34]{CMS:CMS-AN-2016-036}: ca. 10\%

Scale Factor uncert 2016 \cite[Fig. 73]{CMS:CMS-AN-2017-018}: ca. 30\%
Here: 30\%
Reason: high lumi -> bad tracking performance. Source: Erik Butz, "Mitigation of the Strip Tracker Dynamic Inefficiency (previously known as HIP)", August, 2016.


% !TeX spellcheck = en_US
% !TeX encoding = UTF-8
% !TeX root = ../document.tex

%\chapter*{Abstract}
{\usekomafont{chapter}Abstract}
\chapterheadendvskip

In 2015 and 2016, the \acs{CMS} detector recorded proton-proton collisions at an unprecedented center of mass energy of $\sqrt{s} = \SI{13}{\TeV}$. The \acf{MUSiC} provides an automated search for various possible signatures of new physics in these data.

In a three step process, \acs{MUSiC} first classifies events according to the physics content of the final state, searches a set of kinematic distributions for the most significant deviation between \acl{SM} \acl{MC} simulations and observed data and finally applies a statistical hypothesis test to draw conclusions about indications of new physics in the observed dataset.

In this thesis, the discovery potential towards new physics is assessed. For this purpose, a quantification of the test power is defined. Subsequently, the framework is applied to simulated events of four benchmark models for new physics. Alongside the discovery potential towards these theories, the influence of several existing and newly introduced features and parameters on the sensitivity is measured.

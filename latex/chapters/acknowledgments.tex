% !TeX spellcheck = de_DE
% !TeX encoding = UTF-8
% !TeX root = ../document.tex

\thispagestyle{plain}
{\usekomafont{chapter}Danksagung}
\chapterheadendvskip

Diese Masterarbeit wäre nicht möglich gewesen ohne viele Menschen, die mich während des Studiums und der Arbeit unterstützt haben.

Für die Chance, dieses Thema im Rahmen der \ac{MUSiC}-Analyse zu bearbeiten und damit direkt in der Forschung mitzuwirken, für die Unterstützung während der Entwicklung und für die Korrektur der Arbeit danke ich ganz besonders Herrn Professor Thomas Hebbeker.
Des weiteren danke ich Dr.~Arnd Meyer dafür, dass er über viele Jahre einen Überblick über die \ac{MUSiC}-Analyse behalten hat, und mit seinem Expertenwissen bei Fragen zur Seite stand.

Ein ganz besonderer Dank gilt Tobias Pook, als Betreuer, \ac{MUSiC}-Kollege und für das Korrekturlesen der Arbeit. Er hat die Analyse (\ac{MUSiC} bei \SI{13}{\TeV}) mit hervorragendem physikalischen Gespür und Erfahrung in Softwareentwicklung geleitet. Mit seinem Blick für das Essentielle hat er mir stets geholfen, der Arbeit einen Fokus zu geben.

Für die lockere, professionelle Arbeitsatmosphäre danke ich auch der restlichen \ac{MUSiC}-Gruppe, bestehend aus Jonas Roemer, der die \ac{MUSiC}-Klassifikation maßgeblich weiterentwickelt hat, Simon Knutzen, mit dem ich stets komplizierte Statistikfragen diskutieren konnte und Debbie Duchardt, die mit der Analyse des \ac{CMS}-Datensatzes von 2012 eine wichtige Grundlage gesetzt hat.

Ich danke außerdem den anderen Wissenschaftlern der \ac{CMS}-Analysegruppe des III. Physikalischen Instituts A, mit denen ich eng zusammengearbeitet habe. Hierbei sind nicht nur fruchtbare Forschungsergebnisse, sondern auch Freundschaften entstanden.

Professor Martin Erdmann möchte ich nicht nur als Zweitkorrektor dieser Arbeit Dank aussprechen, sondern auch für seinen Einsatz für die Erasmus-Physikstudenten der RWTH. Er hat es mir ermöglicht, zwei Semester meines Masterstudiums an der KTH Royal Institute of Technology in Stockholm zu absolvieren.

Weiterhin möchte ich Dr. Markus Merschmeyer für die Stelle als studentische Hilfskraft danken, durch welche ich meine Kenntnisse der Webentwicklung ausbauen und mein Masterstudium mitfinanzieren konnte.

Zu guter Letzt danke ich meinen Eltern, meinem Bruder und meiner Oma, die mich während meines gesamten Studiums herzlich unterstützt haben, meiner Freundin, die während der stressigeren letzten Monate zu mir stand und meinen Freunden, aus dem Studium, meiner Laufgruppe und dem Aachener Studentenorchester.
